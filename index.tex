% Options for packages loaded elsewhere
\PassOptionsToPackage{unicode}{hyperref}
\PassOptionsToPackage{hyphens}{url}
\PassOptionsToPackage{dvipsnames,svgnames,x11names}{xcolor}
%
\documentclass[
  letterpaper,
  DIV=11,
  numbers=noendperiod]{scrreprt}

\usepackage{amsmath,amssymb}
\usepackage{iftex}
\ifPDFTeX
  \usepackage[T1]{fontenc}
  \usepackage[utf8]{inputenc}
  \usepackage{textcomp} % provide euro and other symbols
\else % if luatex or xetex
  \usepackage{unicode-math}
  \defaultfontfeatures{Scale=MatchLowercase}
  \defaultfontfeatures[\rmfamily]{Ligatures=TeX,Scale=1}
\fi
\usepackage{lmodern}
\ifPDFTeX\else  
    % xetex/luatex font selection
\fi
% Use upquote if available, for straight quotes in verbatim environments
\IfFileExists{upquote.sty}{\usepackage{upquote}}{}
\IfFileExists{microtype.sty}{% use microtype if available
  \usepackage[]{microtype}
  \UseMicrotypeSet[protrusion]{basicmath} % disable protrusion for tt fonts
}{}
\makeatletter
\@ifundefined{KOMAClassName}{% if non-KOMA class
  \IfFileExists{parskip.sty}{%
    \usepackage{parskip}
  }{% else
    \setlength{\parindent}{0pt}
    \setlength{\parskip}{6pt plus 2pt minus 1pt}}
}{% if KOMA class
  \KOMAoptions{parskip=half}}
\makeatother
\usepackage{xcolor}
\setlength{\emergencystretch}{3em} % prevent overfull lines
\setcounter{secnumdepth}{5}
% Make \paragraph and \subparagraph free-standing
\makeatletter
\ifx\paragraph\undefined\else
  \let\oldparagraph\paragraph
  \renewcommand{\paragraph}{
    \@ifstar
      \xxxParagraphStar
      \xxxParagraphNoStar
  }
  \newcommand{\xxxParagraphStar}[1]{\oldparagraph*{#1}\mbox{}}
  \newcommand{\xxxParagraphNoStar}[1]{\oldparagraph{#1}\mbox{}}
\fi
\ifx\subparagraph\undefined\else
  \let\oldsubparagraph\subparagraph
  \renewcommand{\subparagraph}{
    \@ifstar
      \xxxSubParagraphStar
      \xxxSubParagraphNoStar
  }
  \newcommand{\xxxSubParagraphStar}[1]{\oldsubparagraph*{#1}\mbox{}}
  \newcommand{\xxxSubParagraphNoStar}[1]{\oldsubparagraph{#1}\mbox{}}
\fi
\makeatother

\usepackage{color}
\usepackage{fancyvrb}
\newcommand{\VerbBar}{|}
\newcommand{\VERB}{\Verb[commandchars=\\\{\}]}
\DefineVerbatimEnvironment{Highlighting}{Verbatim}{commandchars=\\\{\}}
% Add ',fontsize=\small' for more characters per line
\usepackage{framed}
\definecolor{shadecolor}{RGB}{241,243,245}
\newenvironment{Shaded}{\begin{snugshade}}{\end{snugshade}}
\newcommand{\AlertTok}[1]{\textcolor[rgb]{0.68,0.00,0.00}{#1}}
\newcommand{\AnnotationTok}[1]{\textcolor[rgb]{0.37,0.37,0.37}{#1}}
\newcommand{\AttributeTok}[1]{\textcolor[rgb]{0.40,0.45,0.13}{#1}}
\newcommand{\BaseNTok}[1]{\textcolor[rgb]{0.68,0.00,0.00}{#1}}
\newcommand{\BuiltInTok}[1]{\textcolor[rgb]{0.00,0.23,0.31}{#1}}
\newcommand{\CharTok}[1]{\textcolor[rgb]{0.13,0.47,0.30}{#1}}
\newcommand{\CommentTok}[1]{\textcolor[rgb]{0.37,0.37,0.37}{#1}}
\newcommand{\CommentVarTok}[1]{\textcolor[rgb]{0.37,0.37,0.37}{\textit{#1}}}
\newcommand{\ConstantTok}[1]{\textcolor[rgb]{0.56,0.35,0.01}{#1}}
\newcommand{\ControlFlowTok}[1]{\textcolor[rgb]{0.00,0.23,0.31}{\textbf{#1}}}
\newcommand{\DataTypeTok}[1]{\textcolor[rgb]{0.68,0.00,0.00}{#1}}
\newcommand{\DecValTok}[1]{\textcolor[rgb]{0.68,0.00,0.00}{#1}}
\newcommand{\DocumentationTok}[1]{\textcolor[rgb]{0.37,0.37,0.37}{\textit{#1}}}
\newcommand{\ErrorTok}[1]{\textcolor[rgb]{0.68,0.00,0.00}{#1}}
\newcommand{\ExtensionTok}[1]{\textcolor[rgb]{0.00,0.23,0.31}{#1}}
\newcommand{\FloatTok}[1]{\textcolor[rgb]{0.68,0.00,0.00}{#1}}
\newcommand{\FunctionTok}[1]{\textcolor[rgb]{0.28,0.35,0.67}{#1}}
\newcommand{\ImportTok}[1]{\textcolor[rgb]{0.00,0.46,0.62}{#1}}
\newcommand{\InformationTok}[1]{\textcolor[rgb]{0.37,0.37,0.37}{#1}}
\newcommand{\KeywordTok}[1]{\textcolor[rgb]{0.00,0.23,0.31}{\textbf{#1}}}
\newcommand{\NormalTok}[1]{\textcolor[rgb]{0.00,0.23,0.31}{#1}}
\newcommand{\OperatorTok}[1]{\textcolor[rgb]{0.37,0.37,0.37}{#1}}
\newcommand{\OtherTok}[1]{\textcolor[rgb]{0.00,0.23,0.31}{#1}}
\newcommand{\PreprocessorTok}[1]{\textcolor[rgb]{0.68,0.00,0.00}{#1}}
\newcommand{\RegionMarkerTok}[1]{\textcolor[rgb]{0.00,0.23,0.31}{#1}}
\newcommand{\SpecialCharTok}[1]{\textcolor[rgb]{0.37,0.37,0.37}{#1}}
\newcommand{\SpecialStringTok}[1]{\textcolor[rgb]{0.13,0.47,0.30}{#1}}
\newcommand{\StringTok}[1]{\textcolor[rgb]{0.13,0.47,0.30}{#1}}
\newcommand{\VariableTok}[1]{\textcolor[rgb]{0.07,0.07,0.07}{#1}}
\newcommand{\VerbatimStringTok}[1]{\textcolor[rgb]{0.13,0.47,0.30}{#1}}
\newcommand{\WarningTok}[1]{\textcolor[rgb]{0.37,0.37,0.37}{\textit{#1}}}

\providecommand{\tightlist}{%
  \setlength{\itemsep}{0pt}\setlength{\parskip}{0pt}}\usepackage{longtable,booktabs,array}
\usepackage{calc} % for calculating minipage widths
% Correct order of tables after \paragraph or \subparagraph
\usepackage{etoolbox}
\makeatletter
\patchcmd\longtable{\par}{\if@noskipsec\mbox{}\fi\par}{}{}
\makeatother
% Allow footnotes in longtable head/foot
\IfFileExists{footnotehyper.sty}{\usepackage{footnotehyper}}{\usepackage{footnote}}
\makesavenoteenv{longtable}
\usepackage{graphicx}
\makeatletter
\def\maxwidth{\ifdim\Gin@nat@width>\linewidth\linewidth\else\Gin@nat@width\fi}
\def\maxheight{\ifdim\Gin@nat@height>\textheight\textheight\else\Gin@nat@height\fi}
\makeatother
% Scale images if necessary, so that they will not overflow the page
% margins by default, and it is still possible to overwrite the defaults
% using explicit options in \includegraphics[width, height, ...]{}
\setkeys{Gin}{width=\maxwidth,height=\maxheight,keepaspectratio}
% Set default figure placement to htbp
\makeatletter
\def\fps@figure{htbp}
\makeatother
% definitions for citeproc citations
\NewDocumentCommand\citeproctext{}{}
\NewDocumentCommand\citeproc{mm}{%
  \begingroup\def\citeproctext{#2}\cite{#1}\endgroup}
\makeatletter
 % allow citations to break across lines
 \let\@cite@ofmt\@firstofone
 % avoid brackets around text for \cite:
 \def\@biblabel#1{}
 \def\@cite#1#2{{#1\if@tempswa , #2\fi}}
\makeatother
\newlength{\cslhangindent}
\setlength{\cslhangindent}{1.5em}
\newlength{\csllabelwidth}
\setlength{\csllabelwidth}{3em}
\newenvironment{CSLReferences}[2] % #1 hanging-indent, #2 entry-spacing
 {\begin{list}{}{%
  \setlength{\itemindent}{0pt}
  \setlength{\leftmargin}{0pt}
  \setlength{\parsep}{0pt}
  % turn on hanging indent if param 1 is 1
  \ifodd #1
   \setlength{\leftmargin}{\cslhangindent}
   \setlength{\itemindent}{-1\cslhangindent}
  \fi
  % set entry spacing
  \setlength{\itemsep}{#2\baselineskip}}}
 {\end{list}}
\usepackage{calc}
\newcommand{\CSLBlock}[1]{\hfill\break\parbox[t]{\linewidth}{\strut\ignorespaces#1\strut}}
\newcommand{\CSLLeftMargin}[1]{\parbox[t]{\csllabelwidth}{\strut#1\strut}}
\newcommand{\CSLRightInline}[1]{\parbox[t]{\linewidth - \csllabelwidth}{\strut#1\strut}}
\newcommand{\CSLIndent}[1]{\hspace{\cslhangindent}#1}

\KOMAoption{captions}{tableheading}
\makeatletter
\@ifpackageloaded{tcolorbox}{}{\usepackage[skins,breakable]{tcolorbox}}
\@ifpackageloaded{fontawesome5}{}{\usepackage{fontawesome5}}
\definecolor{quarto-callout-color}{HTML}{909090}
\definecolor{quarto-callout-note-color}{HTML}{0758E5}
\definecolor{quarto-callout-important-color}{HTML}{CC1914}
\definecolor{quarto-callout-warning-color}{HTML}{EB9113}
\definecolor{quarto-callout-tip-color}{HTML}{00A047}
\definecolor{quarto-callout-caution-color}{HTML}{FC5300}
\definecolor{quarto-callout-color-frame}{HTML}{acacac}
\definecolor{quarto-callout-note-color-frame}{HTML}{4582ec}
\definecolor{quarto-callout-important-color-frame}{HTML}{d9534f}
\definecolor{quarto-callout-warning-color-frame}{HTML}{f0ad4e}
\definecolor{quarto-callout-tip-color-frame}{HTML}{02b875}
\definecolor{quarto-callout-caution-color-frame}{HTML}{fd7e14}
\makeatother
\makeatletter
\@ifpackageloaded{bookmark}{}{\usepackage{bookmark}}
\makeatother
\makeatletter
\@ifpackageloaded{caption}{}{\usepackage{caption}}
\AtBeginDocument{%
\ifdefined\contentsname
  \renewcommand*\contentsname{Table of contents}
\else
  \newcommand\contentsname{Table of contents}
\fi
\ifdefined\listfigurename
  \renewcommand*\listfigurename{List of Figures}
\else
  \newcommand\listfigurename{List of Figures}
\fi
\ifdefined\listtablename
  \renewcommand*\listtablename{List of Tables}
\else
  \newcommand\listtablename{List of Tables}
\fi
\ifdefined\figurename
  \renewcommand*\figurename{Figure}
\else
  \newcommand\figurename{Figure}
\fi
\ifdefined\tablename
  \renewcommand*\tablename{Table}
\else
  \newcommand\tablename{Table}
\fi
}
\@ifpackageloaded{float}{}{\usepackage{float}}
\floatstyle{ruled}
\@ifundefined{c@chapter}{\newfloat{codelisting}{h}{lop}}{\newfloat{codelisting}{h}{lop}[chapter]}
\floatname{codelisting}{Listing}
\newcommand*\listoflistings{\listof{codelisting}{List of Listings}}
\makeatother
\makeatletter
\makeatother
\makeatletter
\@ifpackageloaded{caption}{}{\usepackage{caption}}
\@ifpackageloaded{subcaption}{}{\usepackage{subcaption}}
\makeatother

\ifLuaTeX
  \usepackage{selnolig}  % disable illegal ligatures
\fi
\usepackage{bookmark}

\IfFileExists{xurl.sty}{\usepackage{xurl}}{} % add URL line breaks if available
\urlstyle{same} % disable monospaced font for URLs
\hypersetup{
  pdftitle={European Financial Planning Association   Certificación EFA™ (European Financial Advisor)},
  pdfauthor={Alberto Bernat},
  colorlinks=true,
  linkcolor={blue},
  filecolor={Maroon},
  citecolor={Blue},
  urlcolor={Blue},
  pdfcreator={LaTeX via pandoc}}


\title{\emph{European Financial Planning Association} Certificación EFA™
(European Financial Advisor)}
\usepackage{etoolbox}
\makeatletter
\providecommand{\subtitle}[1]{% add subtitle to \maketitle
  \apptocmd{\@title}{\par {\large #1 \par}}{}{}
}
\makeatother
\subtitle{ \textbf{Ejercicios y exámenes resueltos}}
\author{Alberto Bernat}
\date{2025-03-25}

\begin{document}
\maketitle

\renewcommand*\contentsname{Table of contents}
{
\hypersetup{linkcolor=}
\setcounter{tocdepth}{2}
\tableofcontents
}

\bookmarksetup{startatroot}

\chapter*{Preface}\label{preface}
\addcontentsline{toc}{chapter}{Preface}

\markboth{Preface}{Preface}

\includegraphics{images/EFA_logo.jpg}

\section*{Propuesta formativa 🎓}\label{propuesta-formativa}
\addcontentsline{toc}{section}{Propuesta formativa 🎓}

\markright{Propuesta formativa 🎓}

Esta propuesta formativa está diseñada para preparar a un grupo de 5
personas para la certificación \textbf{EFA Nivel II (NII) de EFPA},
abarcando todos los contenidos avanzados (excluyendo el módulo 4 de
seguros y el módulo 9 de cumplimiento normativo y regulador).\\
El programa consta de 12 sesiones de 1,5 horas cada una, impartidas
todos los martes, de \textbf{10:00 a 11:30}, a partir del próximo
martes.

\begin{itemize}
\tightlist
\item
  \textbf{Sesión 1:} Prueba de nivel inicial y repaso del nivel I.\\
\item
  \textbf{Sesiones 2 a 11:} Sesiones centrales que cubrirán los
  contenidos del nivel II.\\
\item
  \textbf{Sesión 6:} Evaluación del progreso y determinación para acudir
  a la convocatoria de junio (se habrán visto lo más exigente del
  examen).\\
\item
  \textbf{Sesión 12:} Simulación de examen corregido y comentado en
  clase.
\end{itemize}

El precio total del programa es de \textbf{2.500 €}. Una vez formalizado
el pago, se enviará una prueba de nivel inicial en los próximos días y
la formación dará comienzo el próximo martes.

\subsection*{📅 Calendario de sesiones}\label{calendario-de-sesiones}
\addcontentsline{toc}{subsection}{📅 Calendario de sesiones}

\begin{longtable}[]{@{}
  >{\raggedright\arraybackslash}p{(\columnwidth - 6\tabcolsep) * \real{0.1250}}
  >{\raggedright\arraybackslash}p{(\columnwidth - 6\tabcolsep) * \real{0.1250}}
  >{\raggedright\arraybackslash}p{(\columnwidth - 6\tabcolsep) * \real{0.1250}}
  >{\raggedright\arraybackslash}p{(\columnwidth - 6\tabcolsep) * \real{0.6250}}@{}}
\toprule\noalign{}
\begin{minipage}[b]{\linewidth}\raggedright
Sesión
\end{minipage} & \begin{minipage}[b]{\linewidth}\raggedright
Fecha
\end{minipage} & \begin{minipage}[b]{\linewidth}\raggedright
Horario
\end{minipage} & \begin{minipage}[b]{\linewidth}\raggedright
Contenido/Objetivo
\end{minipage} \\
\midrule\noalign{}
\endhead
\bottomrule\noalign{}
\endlastfoot
1 & 25 de marzo 2025 & 10:00 -- 11:30 & Prueba de nivel inicial y repaso
del nivel I. 😊 \\
2 & 1 de abril 2025 & 10:00 -- 11:30 & \textbf{Módulo 1 -- Parte I:
Renta fija}Curva de tipos, teorías de la ETTI y gestión del riesgo
(sensibilidad, duración, duración corregida, inmunización). \\
3 & 8 de abril 2025 & 10:00 -- 11:30 & \textbf{Módulo 1 -- Parte II:
Renta variable}Análisis fundamental, ratios bursátiles (Earning Yield
Gap, precio cash flow, precio sobre valor contable, ROA, ROE),
valoración y análisis técnico. \\
4 & 15 de abril 2025 & 10:00 -- 11:30 & \textbf{Módulo 2: Fondos y
sociedades de inversión mobiliaria}.Fondos de inversión libre (hedge
funds), estilos de gestión y análisis/selección de fondos. \\
5 & 22 de abril 2025 & 10:00 -- 11:30 & \textbf{Módulo 3: Gestión de
carteras}- Riesgo y marco de rendimiento (riesgo de cartera,
diversificación).- Mercados de capital eficientes (concepto, hipótesis y
anomalías).- Teoría de carteras (fundamentos, selección de cartera
óptima, modelo de Sharpe, CAPM y otros modelos).- Proceso de asignación
de activos y medición/atribución de resultados (incluyendo normas
GIPS). \\
6 & 29 de abril 2025 & 10:00 -- 11:30 & \textbf{Módulo 8:
Fiscalidad}Aplicación de la normativa fiscal y estrategias de
optimización tributaria. \\
6.5 & 29 de abril 2025 & 11:30 -- 11:40 & \textbf{Evaluación para la
convocatoria de junio}Revisión y valoración de la preparación del
grupo. \\
7 & 6 de mayo 2025 & 10:00 -- 11:30 & \textbf{Módulo 5: Pensiones y
planificación de jubilación}Fundamentos para planificar la jubilación:
definición de necesidades, coberturas de pensiones públicas,
establecimiento de prioridades financieras, inicio del ahorro y análisis
de ingresos y gastos. \\
8 & 13 de mayo 2025 & 10:00 -- 11:30 & \textbf{Módulo 6: Inversión
inmobiliaria}Análisis de características, clasificación, vehículos y
planificación en el sector inmobiliario. \\
9 & 20 de mayo 2025 & 10:00 -- 11:30 & \textbf{Módulo 7: Crédito y
financiación}Análisis de tipos de préstamos, evaluación del riesgo de
crédito y métodos de financiación. \\
10 & 27 de mayo 2025 & 10:00 -- 11:30 & \textbf{Módulo 10: Asesoramiento
y planificación financiera}Estrategias de asesoramiento integral, diseño
de planes financieros y seguimiento. \\
11 & 3 de junio 2025 & 10:00 -- 11:30 & Repaso integral y ajustes de
contenidos.Evaluación final para determinar la asistencia a la
convocatoria de septiembre. \\
12 & 10 de junio 2025 & 10:00 -- 11:30 & Simulación de examen corregido
y comentado en clase.Evaluación final de la formación. \\
\end{longtable}

\subsection*{Estructura del examen EFA NII
📊}\label{estructura-del-examen-efa-nii}
\addcontentsline{toc}{subsection}{Estructura del examen EFA NII 📊}

La Asociación Europea no facilita una estructura detallada y concreta de
la ponderación relativa de los contenidos en el examen EFA NII. No
obstante, se acepta que aproximadamente el \textbf{60\%} del temario
corresponde al nivel uno (conocimientos básicos) y el \textbf{40\%}
restante se destina a los aspectos avanzados o de nivel dos.

La siguiente tabla, que muestra la ponderación aproximada de cada módulo
en el examen EFA completo, nos permite visualizar de forma gráfica cómo
se distribuyen los contenidos en términos de peso:

\begin{longtable}[]{@{}
  >{\raggedright\arraybackslash}p{(\columnwidth - 4\tabcolsep) * \real{0.3836}}
  >{\raggedright\arraybackslash}p{(\columnwidth - 4\tabcolsep) * \real{0.4110}}
  >{\raggedright\arraybackslash}p{(\columnwidth - 4\tabcolsep) * \real{0.2055}}@{}}
\toprule\noalign{}
\begin{minipage}[b]{\linewidth}\raggedright
Módulo
\end{minipage} & \begin{minipage}[b]{\linewidth}\raggedright
Contenido
\end{minipage} & \begin{minipage}[b]{\linewidth}\raggedright
Peso (\%)
\end{minipage} \\
\midrule\noalign{}
\endhead
\bottomrule\noalign{}
\endlastfoot
Módulo 1: Instrumentos y mercados financieros & Instrumentos y mercados
financieros & 25,00\% \\
Módulo 2: Fondos y sociedades de inversión mobiliaria & Fondos y
sociedades de inversión mobiliaria & 10,00\% \\
Módulo 3: Gestión de carteras & Gestión de carteras & 17,50\% \\
Módulo 4: Seguros & Seguros & 7,50\% \\
Módulo 5: Pensiones y planificación de jubilación & Pensiones y
planificación de jubilación & 5,00\% \\
Módulo 6: Inversión inmobiliaria & Inversión inmobiliaria & 5,00\% \\
Módulo 7: Crédito y financiación & Crédito y financiación & 5,00\% \\
Módulo 8: Fiscalidad & Fiscalidad & 10,00\% \\
Módulo 9: Cumplimiento normativo y regulador & Cumplimiento normativo y
regulador & 7,50\% \\
Módulo 10: Asesoramiento y planificación financiera & Asesoramiento y
planificación financiera & 7,50\% \\
\end{longtable}

\emph{Nota:} En la formación \textbf{EFA Nivel II (NII)} se excluyen los
contenidos del Módulo 4 (Seguros) y del Módulo 9 (Cumplimiento normativo
y regulador). Así, el 40\% de los contenidos avanzados se orienta a los
módulos restantes, lo que nos ayuda a focalizar nuestra preparación para
la parte de nivel dos del examen.

\subsection*{Detalles adicionales 💰}\label{detalles-adicionales}
\addcontentsline{toc}{subsection}{Detalles adicionales 💰}

\begin{itemize}
\tightlist
\item
  \textbf{Precio total:} 2.500 €\\
\item
  \textbf{Modalidad:} Sesiones online, todos los martes de 10:00 a
  11:30\\
\item
  \textbf{Grupo:} 5 personas\\
\item
  Los alumnos recibirán una prueba de nivel inicial en los próximos
  días, una vez formalizado el pago.\\
\item
  La formación dará comienzo el próximo martes, 25 de marzo de 2025.
\end{itemize}

\begin{center}\rule{0.5\linewidth}{0.5pt}\end{center}

\section*{Conclusión 🔔}\label{conclusiuxf3n}
\addcontentsline{toc}{section}{Conclusión 🔔}

\markright{Conclusión 🔔}

Esta propuesta está diseñada para ofrecer una preparación intensiva que
cubra los contenidos avanzados del examen \textbf{EFA (EFA NII)}. Se
opta por iniciar la formación de forma parcial, acreditando primero la
parte común y más importante del contenido (aproximadamente el 70\% del
temario) y, posteriormente, mediante una valoración conjunta entre los
alumnos y el profesor (30\% restante), acceder al EFA NII.

La evaluación intermedia en la sesión 6 permitirá, de forma conjunta,
valorar si el grupo está en condiciones de afrontar la convocatoria de
junio (26 de junio) o, en su defecto, optar por la de septiembre (25 de
septiembre).

Nuestra estrategia formativa se orienta a evaluar la preparación del
grupo en dos momentos clave:

\begin{itemize}
\item
  \textbf{Convocatoria de junio (26 de junio):}\\
  Las primeras seis sesiones serán determinantes para valorar el nivel
  del grupo y decidir si se acude a esta convocatoria. En la sesión 6
  (junto con la evaluación intermedia) se analizarán los contenidos más
  exigentes del examen. 🚀
\item
  \textbf{Convocatoria de septiembre (25 de septiembre):}\\
  En la sesión 10 se realizará un repaso integral que, en función del
  rendimiento del grupo, permitirá decidir si se apunta a la
  convocatoria de septiembre. 📆
\end{itemize}

Este enfoque garantiza que la formación se ajuste a las necesidades del
grupo, permitiendo acreditar inicialmente la parte común del temario y,
mediante la evaluación conjunta, determinar la mejor opción de
convocatoria para maximizar las posibilidades de éxito en el examen EFA
NII.

\begin{quote}
\textbf{Continuidad formativa:}\\
Los candidatos que, habiendo consumido las 12 sesiones, deseen continuar
con la formación, podrán negociar su continuidad siempre que el grupo
mínimo sea de 3 personas. En este caso, se aplicará un \textbf{descuento
del 20\%} sobre el precio total del programa. 💼🎉
\end{quote}

\part{SESIONES (vídeos y materiales)}

\chapter{Sesión 1: Repaso del Nivel I y prueba
inicial}\label{sesiuxf3n-1-repaso-del-nivel-i-y-prueba-inicial}

\begin{tcolorbox}[enhanced jigsaw, leftrule=.75mm, breakable, opacitybacktitle=0.6, coltitle=black, rightrule=.15mm, colframe=quarto-callout-note-color-frame, titlerule=0mm, toprule=.15mm, colback=white, arc=.35mm, title=\textcolor{quarto-callout-note-color}{\faInfo}\hspace{0.5em}{Material de la sesión}, toptitle=1mm, bottomrule=.15mm, colbacktitle=quarto-callout-note-color!10!white, left=2mm, opacityback=0, bottomtitle=1mm]

Aquí puedes incluir los vídeos, enlaces y materiales descargables.

\end{tcolorbox}

\chapter{Sesión 2: Módulo 1 -- Parte I: Renta
fija}\label{sesiuxf3n-2-muxf3dulo-1-parte-i-renta-fija}

\begin{tcolorbox}[enhanced jigsaw, leftrule=.75mm, breakable, opacitybacktitle=0.6, coltitle=black, rightrule=.15mm, colframe=quarto-callout-note-color-frame, titlerule=0mm, toprule=.15mm, colback=white, arc=.35mm, title=\textcolor{quarto-callout-note-color}{\faInfo}\hspace{0.5em}{Material de la sesión}, toptitle=1mm, bottomrule=.15mm, colbacktitle=quarto-callout-note-color!10!white, left=2mm, opacityback=0, bottomtitle=1mm]

Aquí puedes incluir los vídeos, enlaces y materiales descargables.

\end{tcolorbox}

\chapter{Sesión 3: Módulo 1 -- Parte II: Renta
variable}\label{sesiuxf3n-3-muxf3dulo-1-parte-ii-renta-variable}

\begin{tcolorbox}[enhanced jigsaw, leftrule=.75mm, breakable, opacitybacktitle=0.6, coltitle=black, rightrule=.15mm, colframe=quarto-callout-note-color-frame, titlerule=0mm, toprule=.15mm, colback=white, arc=.35mm, title=\textcolor{quarto-callout-note-color}{\faInfo}\hspace{0.5em}{Material de la sesión}, toptitle=1mm, bottomrule=.15mm, colbacktitle=quarto-callout-note-color!10!white, left=2mm, opacityback=0, bottomtitle=1mm]

Aquí puedes incluir los vídeos, enlaces y materiales descargables.

\end{tcolorbox}

\chapter{Sesión 4: Módulo 2 -- Fondos y sociedades de
inversión}\label{sesiuxf3n-4-muxf3dulo-2-fondos-y-sociedades-de-inversiuxf3n}

\begin{tcolorbox}[enhanced jigsaw, leftrule=.75mm, breakable, opacitybacktitle=0.6, coltitle=black, rightrule=.15mm, colframe=quarto-callout-note-color-frame, titlerule=0mm, toprule=.15mm, colback=white, arc=.35mm, title=\textcolor{quarto-callout-note-color}{\faInfo}\hspace{0.5em}{Material de la sesión}, toptitle=1mm, bottomrule=.15mm, colbacktitle=quarto-callout-note-color!10!white, left=2mm, opacityback=0, bottomtitle=1mm]

Aquí puedes incluir los vídeos, enlaces y materiales descargables.

\end{tcolorbox}

\chapter{Sesión 5: Módulo 3 -- Gestión de
carteras}\label{sesiuxf3n-5-muxf3dulo-3-gestiuxf3n-de-carteras}

\begin{tcolorbox}[enhanced jigsaw, leftrule=.75mm, breakable, opacitybacktitle=0.6, coltitle=black, rightrule=.15mm, colframe=quarto-callout-note-color-frame, titlerule=0mm, toprule=.15mm, colback=white, arc=.35mm, title=\textcolor{quarto-callout-note-color}{\faInfo}\hspace{0.5em}{Material de la sesión}, toptitle=1mm, bottomrule=.15mm, colbacktitle=quarto-callout-note-color!10!white, left=2mm, opacityback=0, bottomtitle=1mm]

Aquí puedes incluir los vídeos, enlaces y materiales descargables.

\end{tcolorbox}

\chapter{Sesión 6: Módulo 8 --
Fiscalidad}\label{sesiuxf3n-6-muxf3dulo-8-fiscalidad}

\begin{tcolorbox}[enhanced jigsaw, leftrule=.75mm, breakable, opacitybacktitle=0.6, coltitle=black, rightrule=.15mm, colframe=quarto-callout-note-color-frame, titlerule=0mm, toprule=.15mm, colback=white, arc=.35mm, title=\textcolor{quarto-callout-note-color}{\faInfo}\hspace{0.5em}{Material de la sesión}, toptitle=1mm, bottomrule=.15mm, colbacktitle=quarto-callout-note-color!10!white, left=2mm, opacityback=0, bottomtitle=1mm]

Aquí puedes incluir los vídeos, enlaces y materiales descargables.

\end{tcolorbox}

\chapter{Sesión 6.5: Evaluación para la convocatoria de
junio}\label{sesiuxf3n-6.5-evaluaciuxf3n-para-la-convocatoria-de-junio}

\begin{tcolorbox}[enhanced jigsaw, leftrule=.75mm, breakable, opacitybacktitle=0.6, coltitle=black, rightrule=.15mm, colframe=quarto-callout-note-color-frame, titlerule=0mm, toprule=.15mm, colback=white, arc=.35mm, title=\textcolor{quarto-callout-note-color}{\faInfo}\hspace{0.5em}{Material de la sesión}, toptitle=1mm, bottomrule=.15mm, colbacktitle=quarto-callout-note-color!10!white, left=2mm, opacityback=0, bottomtitle=1mm]

Aquí puedes incluir los vídeos, enlaces y materiales descargables.

\end{tcolorbox}

\chapter{Sesión 7: Módulo 5 -- Pensiones y planificación de
jubilación}\label{sesiuxf3n-7-muxf3dulo-5-pensiones-y-planificaciuxf3n-de-jubilaciuxf3n}

\begin{tcolorbox}[enhanced jigsaw, leftrule=.75mm, breakable, opacitybacktitle=0.6, coltitle=black, rightrule=.15mm, colframe=quarto-callout-note-color-frame, titlerule=0mm, toprule=.15mm, colback=white, arc=.35mm, title=\textcolor{quarto-callout-note-color}{\faInfo}\hspace{0.5em}{Material de la sesión}, toptitle=1mm, bottomrule=.15mm, colbacktitle=quarto-callout-note-color!10!white, left=2mm, opacityback=0, bottomtitle=1mm]

Aquí puedes incluir los vídeos, enlaces y materiales descargables.

\end{tcolorbox}

\chapter{Sesión 8: Módulo 6 -- Inversión
inmobiliaria}\label{sesiuxf3n-8-muxf3dulo-6-inversiuxf3n-inmobiliaria}

\begin{tcolorbox}[enhanced jigsaw, leftrule=.75mm, breakable, opacitybacktitle=0.6, coltitle=black, rightrule=.15mm, colframe=quarto-callout-note-color-frame, titlerule=0mm, toprule=.15mm, colback=white, arc=.35mm, title=\textcolor{quarto-callout-note-color}{\faInfo}\hspace{0.5em}{Material de la sesión}, toptitle=1mm, bottomrule=.15mm, colbacktitle=quarto-callout-note-color!10!white, left=2mm, opacityback=0, bottomtitle=1mm]

Aquí puedes incluir los vídeos, enlaces y materiales descargables.

\end{tcolorbox}

\chapter{Sesión 9: Módulo 7 -- Crédito y
financiación}\label{sesiuxf3n-9-muxf3dulo-7-cruxe9dito-y-financiaciuxf3n}

\begin{tcolorbox}[enhanced jigsaw, leftrule=.75mm, breakable, opacitybacktitle=0.6, coltitle=black, rightrule=.15mm, colframe=quarto-callout-note-color-frame, titlerule=0mm, toprule=.15mm, colback=white, arc=.35mm, title=\textcolor{quarto-callout-note-color}{\faInfo}\hspace{0.5em}{Material de la sesión}, toptitle=1mm, bottomrule=.15mm, colbacktitle=quarto-callout-note-color!10!white, left=2mm, opacityback=0, bottomtitle=1mm]

Aquí puedes incluir los vídeos, enlaces y materiales descargables.

\end{tcolorbox}

\chapter{Sesión 10: Módulo 10 -- Asesoramiento y planificación
financiera}\label{sesiuxf3n-10-muxf3dulo-10-asesoramiento-y-planificaciuxf3n-financiera}

\begin{tcolorbox}[enhanced jigsaw, leftrule=.75mm, breakable, opacitybacktitle=0.6, coltitle=black, rightrule=.15mm, colframe=quarto-callout-note-color-frame, titlerule=0mm, toprule=.15mm, colback=white, arc=.35mm, title=\textcolor{quarto-callout-note-color}{\faInfo}\hspace{0.5em}{Material de la sesión}, toptitle=1mm, bottomrule=.15mm, colbacktitle=quarto-callout-note-color!10!white, left=2mm, opacityback=0, bottomtitle=1mm]

Aquí puedes incluir los vídeos, enlaces y materiales descargables.

\end{tcolorbox}

\chapter{Sesión 11: Repaso integral y evaluación
final}\label{sesiuxf3n-11-repaso-integral-y-evaluaciuxf3n-final}

\begin{tcolorbox}[enhanced jigsaw, leftrule=.75mm, breakable, opacitybacktitle=0.6, coltitle=black, rightrule=.15mm, colframe=quarto-callout-note-color-frame, titlerule=0mm, toprule=.15mm, colback=white, arc=.35mm, title=\textcolor{quarto-callout-note-color}{\faInfo}\hspace{0.5em}{Material de la sesión}, toptitle=1mm, bottomrule=.15mm, colbacktitle=quarto-callout-note-color!10!white, left=2mm, opacityback=0, bottomtitle=1mm]

Aquí puedes incluir los vídeos, enlaces y materiales descargables.

\end{tcolorbox}

\chapter{Sesión 12: Simulación de examen y cierre del
curso}\label{sesiuxf3n-12-simulaciuxf3n-de-examen-y-cierre-del-curso}

\begin{tcolorbox}[enhanced jigsaw, leftrule=.75mm, breakable, opacitybacktitle=0.6, coltitle=black, rightrule=.15mm, colframe=quarto-callout-note-color-frame, titlerule=0mm, toprule=.15mm, colback=white, arc=.35mm, title=\textcolor{quarto-callout-note-color}{\faInfo}\hspace{0.5em}{Material de la sesión}, toptitle=1mm, bottomrule=.15mm, colbacktitle=quarto-callout-note-color!10!white, left=2mm, opacityback=0, bottomtitle=1mm]

Aquí puedes incluir los vídeos, enlaces y materiales descargables.

\end{tcolorbox}

\part{TEST}

\chapter{Introduction}\label{introduction}

This is a book created from markdown and executable code.

See Knuth (1984) for additional discussion of literate programming.

\begin{Shaded}
\begin{Highlighting}[]
\DecValTok{1} \SpecialCharTok{+} \DecValTok{1}
\end{Highlighting}
\end{Shaded}

\begin{verbatim}
[1] 2
\end{verbatim}

\part{EXÁMENES}

\chapter{Introduction}\label{introduction-1}

This is a book created from markdown and executable code.

See Knuth (1984) for additional discussion of literate programming.

\begin{Shaded}
\begin{Highlighting}[]
\DecValTok{1} \SpecialCharTok{+} \DecValTok{1}
\end{Highlighting}
\end{Shaded}

\begin{verbatim}
[1] 2
\end{verbatim}

\phantomsection\label{refs}
\begin{CSLReferences}{1}{0}
\bibitem[\citeproctext]{ref-knuth84}
Knuth, Donald E. 1984. {``Literate Programming.''} \emph{Comput. J.} 27
(2): 97--111. \url{https://doi.org/10.1093/comjnl/27.2.97}.

\end{CSLReferences}




\end{document}
