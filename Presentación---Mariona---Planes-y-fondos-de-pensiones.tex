% Options for packages loaded elsewhere
% Options for packages loaded elsewhere
\PassOptionsToPackage{unicode}{hyperref}
\PassOptionsToPackage{hyphens}{url}
\PassOptionsToPackage{dvipsnames,svgnames,x11names}{xcolor}
%
\documentclass[
  ignorenonframetext,
  aspectratio=54,
  spanish,
]{beamer}
\newif\ifbibliography
\usepackage{pgfpages}
\setbeamertemplate{caption}[numbered]
\setbeamertemplate{caption label separator}{: }
\setbeamercolor{caption name}{fg=normal text.fg}
\beamertemplatenavigationsymbolsempty
% remove section numbering
\setbeamertemplate{part page}{
  \centering
  \begin{beamercolorbox}[sep=16pt,center]{part title}
    \usebeamerfont{part title}\insertpart\par
  \end{beamercolorbox}
}
\setbeamertemplate{section page}{
  \centering
  \begin{beamercolorbox}[sep=12pt,center]{section title}
    \usebeamerfont{section title}\insertsection\par
  \end{beamercolorbox}
}
\setbeamertemplate{subsection page}{
  \centering
  \begin{beamercolorbox}[sep=8pt,center]{subsection title}
    \usebeamerfont{subsection title}\insertsubsection\par
  \end{beamercolorbox}
}
% Prevent slide breaks in the middle of a paragraph
\widowpenalties 1 10000
\raggedbottom
\AtBeginPart{
  \frame{\partpage}
}
\AtBeginSection{
  \ifbibliography
  \else
    \frame{\sectionpage}
  \fi
}
\AtBeginSubsection{
  \frame{\subsectionpage}
}
\usepackage{iftex}
\ifPDFTeX
  \usepackage[T1]{fontenc}
  \usepackage[utf8]{inputenc}
  \usepackage{textcomp} % provide euro and other symbols
\else % if luatex or xetex
  \usepackage{unicode-math} % this also loads fontspec
  \defaultfontfeatures{Scale=MatchLowercase}
  \defaultfontfeatures[\rmfamily]{Ligatures=TeX,Scale=1}
\fi
\usepackage{lmodern}

\usetheme[]{Madrid}
\usecolortheme{seahorse}
\ifPDFTeX\else
  % xetex/luatex font selection
\fi
% Use upquote if available, for straight quotes in verbatim environments
\IfFileExists{upquote.sty}{\usepackage{upquote}}{}
\IfFileExists{microtype.sty}{% use microtype if available
  \usepackage[]{microtype}
  \UseMicrotypeSet[protrusion]{basicmath} % disable protrusion for tt fonts
}{}
\makeatletter
\@ifundefined{KOMAClassName}{% if non-KOMA class
  \IfFileExists{parskip.sty}{%
    \usepackage{parskip}
  }{% else
    \setlength{\parindent}{0pt}
    \setlength{\parskip}{6pt plus 2pt minus 1pt}}
}{% if KOMA class
  \KOMAoptions{parskip=half}}
\makeatother


\usepackage{longtable,booktabs,array}
\usepackage{calc} % for calculating minipage widths
\usepackage{caption}
% Make caption package work with longtable
\makeatletter
\def\fnum@table{\tablename~\thetable}
\makeatother
\usepackage{graphicx}
\makeatletter
\newsavebox\pandoc@box
\newcommand*\pandocbounded[1]{% scales image to fit in text height/width
  \sbox\pandoc@box{#1}%
  \Gscale@div\@tempa{\textheight}{\dimexpr\ht\pandoc@box+\dp\pandoc@box\relax}%
  \Gscale@div\@tempb{\linewidth}{\wd\pandoc@box}%
  \ifdim\@tempb\p@<\@tempa\p@\let\@tempa\@tempb\fi% select the smaller of both
  \ifdim\@tempa\p@<\p@\scalebox{\@tempa}{\usebox\pandoc@box}%
  \else\usebox{\pandoc@box}%
  \fi%
}
% Set default figure placement to htbp
\def\fps@figure{htbp}
\makeatother



\ifLuaTeX
\usepackage[bidi=basic]{babel}
\else
\usepackage[bidi=default]{babel}
\fi
% get rid of language-specific shorthands (see #6817):
\let\LanguageShortHands\languageshorthands
\def\languageshorthands#1{}


\setlength{\emergencystretch}{3em} % prevent overfull lines

\providecommand{\tightlist}{%
  \setlength{\itemsep}{0pt}\setlength{\parskip}{0pt}}



 


\usepackage{graphicx}
\usepackage{amssymb}
\usepackage{textcomp}
\definecolor{MediolanumBlueLight}{HTML}{1e96d7}
\definecolor{MediolanumBlueDark}{HTML}{192d6e}
\definecolor{EFPAYellow}{RGB}{255,204,0}
\setbeamercolor{title}{fg=MediolanumBlueDark}
\setbeamercolor{frametitle}{fg=MediolanumBlueLight}
\setbeamercolor{structure}{fg=EFPAYellow}
\setbeamercolor{section in toc}{fg=MediolanumBlueDark}
\setbeamercolor{footline}{fg=MediolanumBlueDark, bg=white}
\setbeamercolor{block title}{fg=white, bg=MediolanumBlueDark}
\setbeamercolor{block body}{fg=black, bg=MediolanumBlueLight!20}
\setbeamercolor{alerted text}{fg=EFPAYellow}
\setbeamercolor{item}{fg=MediolanumBlueDark}
\setbeamercolor{subitem}{fg=MediolanumBlueLight}
\setbeamercolor{subsubitem}{fg=EFPAYellow}
\setbeamercolor{caption name}{fg=MediolanumBlueDark}
\setbeamercolor{button}{bg=MediolanumBlueDark, fg=white}
\setbeamercolor{button border}{bg=EFPAYellow, fg=EFPAYellow}
\setbeamercolor{button text}{fg=white}
\setbeamercolor{navigation symbols}{fg=MediolanumBlueLight}
\setbeamercolor{palette primary}{bg=MediolanumBlueDark, fg=white}
\setbeamercolor{palette secondary}{bg=MediolanumBlueLight, fg=white}
\setbeamercolor{palette tertiary}{bg=EFPAYellow, fg=black}
\setbeamercolor{palette quaternary}{bg=MediolanumBlueDark, fg=white}
\setbeamercolor{section in head/foot}{bg=MediolanumBlueDark, fg=white}
\setbeamercolor{subsection in head/foot}{bg=MediolanumBlueLight, fg=white}
\setbeamercolor{author in head/foot}{fg=MediolanumBlueDark}
\setbeamercolor{institute in head/foot}{fg=MediolanumBlueLight}
\setbeamercolor{date in head/foot}{fg=EFPAYellow}
\setbeamercolor{title in head/foot}{fg=MediolanumBlueDark}
\setbeamercolor{subtitle}{fg=EFPAYellow}
\setbeamercolor{logo in head/foot}{bg=white}
\setbeamercolor{bibliography entry author}{fg=MediolanumBlueDark}
\setbeamercolor{bibliography entry title}{fg=MediolanumBlueLight}
\setbeamercolor{bibliography entry location}{fg=EFPAYellow}
\setbeamercolor{bibliography entry note}{fg=MediolanumBlueDark}
\setbeamercolor{abstract}{fg=MediolanumBlueDark, bg=MediolanumBlueLight!20}
\setbeamercolor{abstract title}{fg=white, bg=MediolanumBlueDark}
\setbeamercolor{verse}{fg=MediolanumBlueDark, bg=MediolanumBlueLight!20}
\setbeamercolor{quotation}{fg=MediolanumBlueDark, bg=MediolanumBlueLight!20}
\setbeamercolor{quote}{fg=MediolanumBlueDark, bg=MediolanumBlueLight!20}
\setbeamercolor{description item}{fg=MediolanumBlueDark}
\setbeamercolor{description item label}{fg=MediolanumBlueLight}
\setbeamercolor{description item body}{fg=EFPAYellow}
\setbeamercolor{enumerate item}{fg=MediolanumBlueDark}
\setbeamercolor{enumerate subitem}{fg=MediolanumBlueLight}
\setbeamercolor{enumerate subsubitem}{fg=EFPAYellow}
\setbeamercolor{itemize item}{fg=MediolanumBlueDark}
\setbeamercolor{itemize subitem}{fg=MediolanumBlueLight}
\setbeamercolor{itemize subsubitem}{fg=EFPAYellow}
\setbeamercolor{list item}{fg=MediolanumBlueDark}
\makeatletter
\@ifpackageloaded{tcolorbox}{}{\usepackage[skins,breakable]{tcolorbox}}
\@ifpackageloaded{fontawesome5}{}{\usepackage{fontawesome5}}
\definecolor{quarto-callout-color}{HTML}{909090}
\definecolor{quarto-callout-note-color}{HTML}{0758E5}
\definecolor{quarto-callout-important-color}{HTML}{CC1914}
\definecolor{quarto-callout-warning-color}{HTML}{EB9113}
\definecolor{quarto-callout-tip-color}{HTML}{00A047}
\definecolor{quarto-callout-caution-color}{HTML}{FC5300}
\definecolor{quarto-callout-color-frame}{HTML}{acacac}
\definecolor{quarto-callout-note-color-frame}{HTML}{4582ec}
\definecolor{quarto-callout-important-color-frame}{HTML}{d9534f}
\definecolor{quarto-callout-warning-color-frame}{HTML}{f0ad4e}
\definecolor{quarto-callout-tip-color-frame}{HTML}{02b875}
\definecolor{quarto-callout-caution-color-frame}{HTML}{fd7e14}
\makeatother
\makeatletter
\@ifpackageloaded{caption}{}{\usepackage{caption}}
\AtBeginDocument{%
\ifdefined\contentsname
  \renewcommand*\contentsname{Tabla de contenidos}
\else
  \newcommand\contentsname{Tabla de contenidos}
\fi
\ifdefined\listfigurename
  \renewcommand*\listfigurename{Listado de Figuras}
\else
  \newcommand\listfigurename{Listado de Figuras}
\fi
\ifdefined\listtablename
  \renewcommand*\listtablename{Listado de Tablas}
\else
  \newcommand\listtablename{Listado de Tablas}
\fi
\ifdefined\figurename
  \renewcommand*\figurename{Figura}
\else
  \newcommand\figurename{Figura}
\fi
\ifdefined\tablename
  \renewcommand*\tablename{Tabla}
\else
  \newcommand\tablename{Tabla}
\fi
}
\@ifpackageloaded{float}{}{\usepackage{float}}
\floatstyle{ruled}
\@ifundefined{c@chapter}{\newfloat{codelisting}{h}{lop}}{\newfloat{codelisting}{h}{lop}[chapter]}
\floatname{codelisting}{Listado}
\newcommand*\listoflistings{\listof{codelisting}{Listado de Listados}}
\makeatother
\makeatletter
\makeatother
\makeatletter
\@ifpackageloaded{caption}{}{\usepackage{caption}}
\@ifpackageloaded{subcaption}{}{\usepackage{subcaption}}
\makeatother

\usepackage{bookmark}
\IfFileExists{xurl.sty}{\usepackage{xurl}}{} % add URL line breaks if available
\urlstyle{same}
\hypersetup{
  pdftitle={UEMC online},
  pdfauthor={Profesor Alberto Bernat},
  pdflang={es},
  colorlinks=true,
  linkcolor={Maroon},
  filecolor={Maroon},
  citecolor={Blue},
  urlcolor={Blue},
  pdfcreator={LaTeX via pandoc}}



\title{UEMC online}
\subtitle{Planes y fondos de pensiones}
\author{Profesor Alberto Bernat}
\date{27/05/2025}

\begin{document}
\frame{\titlepage}


\begin{frame}{Fondos y Sociedades de Inversión Mobiliaria}
\phantomsection\label{fondos-y-sociedades-de-inversiuxf3n-mobiliaria}
\pandocbounded{\includegraphics[keepaspectratio]{images/EFPA_logo-01.jpg}}
\end{frame}

\begin{frame}{Objetivo de la sesión}
\phantomsection\label{objetivo-de-la-sesiuxf3n}
\begin{itemize}
\tightlist
\item
  Comprender el funcionamiento de los planes y fondos de pensiones como
  herramientas de previsión social.
\item
  Analizar su tratamiento fiscal, contingencias, modalidades y
  planificación de la jubilación.
\item
  Estimar recursos disponibles y necesidades previsionales en la etapa
  pasiva.
\end{itemize}
\end{frame}

\begin{frame}{¿Qué es un plan de pensiones?}
\phantomsection\label{quuxe9-es-un-plan-de-pensiones}
\begin{itemize}
\tightlist
\item
  Instrumento de \textbf{previsión social voluntaria}.
\item
  Constituido mediante \textbf{aportaciones periódicas o puntuales}.
\item
  Solo puede \textbf{hacerse líquido} ante contingencias específicas o
  supuestos legales.
\end{itemize}
\end{frame}

\begin{frame}{Teoría de los tres pilares}
\phantomsection\label{teoruxeda-de-los-tres-pilares}
\begin{enumerate}
\tightlist
\item
  \textbf{Pilar 1}: Sistema público obligatorio (Seguridad Social).
\item
  \textbf{Pilar 2}: Sistema profesional (planes de empleo).
\item
  \textbf{Pilar 3}: Sistema individual (planes personales, seguros,
  etc.).
\end{enumerate}
\end{frame}

\begin{frame}{Principios rectores de los planes de pensiones}
\phantomsection\label{principios-rectores-de-los-planes-de-pensiones}
Todo plan de pensiones debe respetar cinco principios fundamentales:

\begin{itemize}
\tightlist
\item
  \textbf{No discriminación}: los planes deben permitir el acceso a
  todos los trabajadores que cumplan los requisitos objetivos
  establecidos (por ejemplo, una antigüedad mínima en la empresa),
  \textbf{sin establecer diferencias arbitrarias o excluyentes}.
\item
  \textbf{Capitalización}: acumulación de aportaciones para generar una
  prestación futura.
\item
  \textbf{Irrevocabilidad}: las aportaciones no pueden retirarse
  libremente.
\item
  \textbf{Atribución de derechos}: el partícipe acumula derechos sobre
  lo aportado.
\item
  \textbf{Movilidad}: los derechos consolidados pueden trasladarse a
  otro plan.
\end{itemize}

\begin{tcolorbox}[enhanced jigsaw, rightrule=.15mm, colback=white, arc=.35mm, colframe=quarto-callout-note-color-frame, bottomrule=.15mm, left=2mm, toptitle=1mm, colbacktitle=quarto-callout-note-color!10!white, leftrule=.75mm, bottomtitle=1mm, titlerule=0mm, title=\textcolor{quarto-callout-note-color}{\faInfo}\hspace{0.5em}{Nota}, opacityback=0, coltitle=black, toprule=.15mm, opacitybacktitle=0.6, breakable]

Garantizan transparencia, equidad y protección del ahorro a largo plazo.

\end{tcolorbox}
\end{frame}

\begin{frame}{Ejemplo de aplicación: principio de no discriminación}
\phantomsection\label{ejemplo-de-aplicaciuxf3n-principio-de-no-discriminaciuxf3n}
Los planes de pensiones de empleo \textbf{pueden establecer requisitos
objetivos de acceso}, como la antigüedad mínima en la empresa.

\begin{block}{\textbf{Regla normativa:}}
\phantomsection\label{regla-normativa}
\begin{quote}
Si el requisito de antigüedad es \textbf{igual o inferior a 2 años}, no
requiere justificación especial.
\end{quote}

\begin{quote}
Si supera los \textbf{2 años}, el plan debe \textbf{justificarlo
expresamente} (RD 304/2004, art. 6.2).
\end{quote}
\end{block}

\begin{block}{Ejemplo}
\phantomsection\label{ejemplo}
Una empresa establece que sus empleados pueden adherirse al plan de
pensiones de empleo a partir del segundo año de contrato.

\begin{itemize}
\item
  Este criterio es \textbf{objetivo, no discriminatorio} y está
  \textbf{permitido por la normativa}.
\item
  Este tipo de cláusulas garantizan un trato equitativo sin excluir
  injustamente a determinados colectivos.
\end{itemize}
\end{block}
\end{frame}

\begin{frame}{Elementos personales del plan de pensiones}
\phantomsection\label{elementos-personales-del-plan-de-pensiones}
Un plan de pensiones implica a varios actores:

\begin{itemize}
\tightlist
\item
  \textbf{Promotor}: entidad o empresa que impulsa el plan.
\item
  \textbf{Partícipe}: persona que realiza aportaciones (trabajador o
  particular).
\item
  \textbf{Beneficiario}: quien recibe la prestación en caso de
  contingencia.
\item
  \textbf{Entidad gestora}: administra el fondo y toma decisiones de
  inversión.
\item
  \textbf{Entidad depositaria}: custodia los activos financieros del
  fondo.
\end{itemize}

\begin{tcolorbox}[enhanced jigsaw, rightrule=.15mm, colback=white, arc=.35mm, colframe=quarto-callout-note-color-frame, bottomrule=.15mm, left=2mm, toptitle=1mm, colbacktitle=quarto-callout-note-color!10!white, leftrule=.75mm, bottomtitle=1mm, titlerule=0mm, title=\textcolor{quarto-callout-note-color}{\faInfo}\hspace{0.5em}{Nota}, opacityback=0, coltitle=black, toprule=.15mm, opacitybacktitle=0.6, breakable]

Cada uno cumple una función específica dentro del sistema de previsión.

\end{tcolorbox}
\end{frame}

\begin{frame}{Plan vs Fondo de Pensiones}
\phantomsection\label{plan-vs-fondo-de-pensiones}
\begin{itemize}
\tightlist
\item
  \textbf{Plan}: conjunto de derechos y obligaciones.
\item
  \textbf{Fondo}: patrimonio financiero que da soporte a uno o varios
  planes.
\end{itemize}
\end{frame}

\begin{frame}{Clasificación de planes de pensiones}
\phantomsection\label{clasificaciuxf3n-de-planes-de-pensiones}
\textbf{Según el promotor:}

\begin{itemize}
\tightlist
\item
  Individual
\item
  Empleo
\item
  Asociado
\end{itemize}
\end{frame}

\begin{frame}{Clasificación (II): aportaciones y prestaciones}
\phantomsection\label{clasificaciuxf3n-ii-aportaciones-y-prestaciones}
Los \textbf{planes de pensiones} pueden clasificarse, según el criterio
de determinación de los compromisos económicos del plan, en tres
modalidades:

\begin{itemize}
\tightlist
\item
  Aportación definida
\item
  Prestación definida
\item
  Mixto
\end{itemize}
\end{frame}

\begin{frame}{Aportación definida}
\phantomsection\label{aportaciuxf3n-definida}
En esta modalidad, lo que se define desde el inicio es la
\textbf{cuantía de las aportaciones} que realizará el partícipe o el
promotor (o ambos). La \textbf{prestación futura dependerá del capital
acumulado}, es decir, del total de aportaciones más la rentabilidad
generada por el fondo de pensiones.

\begin{itemize}
\tightlist
\item
  Es el sistema más extendido, especialmente en los \textbf{planes del
  sistema individual}.
\item
  El \textbf{riesgo de inversión recae sobre el partícipe}, ya que no se
  garantiza la cuantía final de la prestación.
\end{itemize}

\textbf{Ejemplo 1:}\\
Un partícipe aporta 2.000\,€ anuales a su plan individual durante 30
años. Al llegar a la jubilación, el capital disponible dependerá del
rendimiento obtenido por el fondo, sin que exista compromiso previo
sobre la cuantía de la prestación.
\end{frame}

\begin{frame}{Prestación definida}
\phantomsection\label{prestaciuxf3n-definida}
En esta modalidad, se define de antemano la \textbf{cuantía de la
prestación} que se recibirá al producirse la contingencia (normalmente
la jubilación). En consecuencia, deben realizarse aportaciones
suficientes para garantizar dicha prestación, lo que implica un
\textbf{mayor compromiso por parte del promotor}.

\begin{itemize}
\tightlist
\item
  Son más frecuentes en \textbf{planes de empleo}, en grandes empresas.
\item
  Requieren \textbf{actuarialmente} estimar aportaciones periódicas para
  cumplir con la prestación prevista.
\end{itemize}

\textbf{Ejemplo 2:}\\
Un plan de empresa garantiza a sus empleados una pensión del 70\,\% del
salario medio de los últimos 10 años. Las aportaciones deben ajustarse
periódicamente para asegurar este objetivo.
\end{frame}

\begin{frame}{Plan mixto}
\phantomsection\label{plan-mixto}
Los planes mixtos \textbf{combinan una parte garantizada y otra
variable} según el capital acumulado, siendo más complejos y habituales
en entornos empresariales.

\textbf{Ejemplo 3:}\\
Un plan mixto establece una pensión mínima garantizada de 500\,€/mes
(prestación definida) y un complemento adicional que dependerá del
capital acumulado (aportación definida).

\begin{tcolorbox}[enhanced jigsaw, rightrule=.15mm, colback=white, arc=.35mm, colframe=quarto-callout-note-color-frame, bottomrule=.15mm, left=2mm, toptitle=1mm, colbacktitle=quarto-callout-note-color!10!white, leftrule=.75mm, bottomtitle=1mm, titlerule=0mm, title=\textcolor{quarto-callout-note-color}{\faInfo}\hspace{0.5em}{\textbf{Importante:}\\
}, opacityback=0, coltitle=black, toprule=.15mm, opacitybacktitle=0.6, breakable]

Los \textbf{planes de pensiones del sistema individual} (aquellos
promovidos por entidades financieras y contratados de forma voluntaria
por personas físicas) \textbf{son siempre de modalidad de aportación
definida}, tal como establece la normativa española.\\

\end{tcolorbox}
\end{frame}

\begin{frame}{Clasificación (III): vocación inversora}
\phantomsection\label{clasificaciuxf3n-iii-vocaciuxf3n-inversora}
\begin{itemize}
\tightlist
\item
  Renta fija corto/largo plazo
\item
  Renta variable
\item
  Mixtos
\item
  Garantizados
\end{itemize}
\end{frame}

\begin{frame}{PPES y PEPP}
\phantomsection\label{ppes-y-pepp}
\begin{itemize}
\tightlist
\item
  \textbf{PPES}: Planes de pensiones de empleo simplificados.
\item
  \textbf{PEPP}: Producto paneuropeo de pensiones individuales
  (portabilidad, transparencia, coste limitado).
\end{itemize}
\end{frame}

\begin{frame}{Supervisión y normativa aplicable}
\phantomsection\label{supervisiuxf3n-y-normativa-aplicable}
\begin{itemize}
\tightlist
\item
  Los \textbf{planes y fondos de pensiones} están regulados por la
  \textbf{Ley de Regulación de los Planes y Fondos de Pensiones} y su
  desarrollo reglamentario.
\item
  El organismo responsable de la supervisión y control es la:
\end{itemize}

\begin{block}{Dirección General de Seguros y Fondos de Pensiones
(DGSFP)}
\phantomsection\label{direcciuxf3n-general-de-seguros-y-fondos-de-pensiones-dgsfp}
\begin{itemize}
\tightlist
\item
  Depende del \textbf{Ministerio de Economía, Comercio y Empresa}.
\item
  Supervisa que los planes cumplan:

  \begin{itemize}
  \tightlist
  \item
    Con los \textbf{principios básicos} (no discriminación,
    capitalización, irrevocabilidad, etc.).
  \item
    Con las \textbf{obligaciones de transparencia e información}.
  \item
    Con la adecuada \textbf{gestión de riesgos e inversiones}.
  \end{itemize}
\end{itemize}
\end{block}
\end{frame}

\begin{frame}{Rentabilidad y riesgo}
\phantomsection\label{rentabilidad-y-riesgo}
\begin{itemize}
\tightlist
\item
  Existe una relación directa entre \textbf{riesgo asumido} y
  \textbf{rentabilidad esperada}.
\item
  Cuanto mayor es la rentabilidad potencial, mayor es también el nivel
  de riesgo.
\item
  A largo plazo, la \textbf{volatilidad tiende a suavizarse}, lo que
  permite asumir más riesgo en horizontes amplios.
\end{itemize}
\end{frame}

\begin{frame}{Nivel de riesgo en planes de pensiones individuales}
\phantomsection\label{nivel-de-riesgo-en-planes-de-pensiones-individuales}
\begin{itemize}
\tightlist
\item
  Cada plan o fondo tiene \textbf{asignado un nivel de riesgo global},
  que \textbf{no varía entre partícipes}.
\item
  El partícipe puede elegir el fondo más adecuado a su perfil (prudente,
  moderado, agresivo).
\item
  La clasificación del riesgo sigue una \textbf{escala oficial de 1 a
  7}, en función de la volatilidad esperada.
\end{itemize}

\begin{tcolorbox}[enhanced jigsaw, rightrule=.15mm, colback=white, arc=.35mm, colframe=quarto-callout-note-color-frame, bottomrule=.15mm, left=2mm, toptitle=1mm, colbacktitle=quarto-callout-note-color!10!white, leftrule=.75mm, bottomtitle=1mm, titlerule=0mm, title=\textcolor{quarto-callout-note-color}{\faInfo}\hspace{0.5em}{Nota}, opacityback=0, coltitle=black, toprule=.15mm, opacitybacktitle=0.6, breakable]

Es fundamental que el partícipe seleccione un fondo acorde con su
tolerancia al riesgo y horizonte temporal.

\end{tcolorbox}
\end{frame}

\begin{frame}{Nivel de riesgo en planes de pensiones de empleo}
\phantomsection\label{nivel-de-riesgo-en-planes-de-pensiones-de-empleo}
\begin{itemize}
\tightlist
\item
  El plan de empleo establece una \textbf{estrategia de inversión común
  para todos los trabajadores}.
\item
  El partícipe \textbf{no elige individualmente el perfil de riesgo}.
\item
  La política de inversión la define la empresa junto con la comisión de
  control.
\item
  Todos los partícipes del plan \textbf{comparten el mismo fondo y el
  mismo nivel de riesgo}.
\end{itemize}

\begin{tcolorbox}[enhanced jigsaw, rightrule=.15mm, colback=white, arc=.35mm, colframe=quarto-callout-note-color-frame, bottomrule=.15mm, left=2mm, toptitle=1mm, colbacktitle=quarto-callout-note-color!10!white, leftrule=.75mm, bottomtitle=1mm, titlerule=0mm, title=\textcolor{quarto-callout-note-color}{\faInfo}\hspace{0.5em}{Nota}, opacityback=0, coltitle=black, toprule=.15mm, opacitybacktitle=0.6, breakable]

En los planes de empleo, la inversión es colectiva y no existe
personalización del riesgo para cada trabajador.

\end{tcolorbox}
\end{frame}

\begin{frame}{Contingencias cubiertas}
\phantomsection\label{contingencias-cubiertas}
\begin{itemize}
\tightlist
\item
  Jubilación
\item
  Incapacidad
\item
  Fallecimiento
\item
  Dependencia
\end{itemize}
\end{frame}

\begin{frame}{Supuestos de disposición anticipada}
\phantomsection\label{supuestos-de-disposiciuxf3n-anticipada}
\begin{itemize}
\tightlist
\item
  Enfermedad grave
\item
  Desempleo larga duración
\item
  Aportaciones \textgreater{} 10 años
\end{itemize}
\end{frame}

\begin{frame}{Planificación de la jubilación: ¿por qué es necesaria?}
\phantomsection\label{planificaciuxf3n-de-la-jubilaciuxf3n-por-quuxe9-es-necesaria}
\begin{itemize}
\tightlist
\item
  \textbf{Caída de ingresos} al pasar a clase pasiva.
\item
  \textbf{Riesgo de longevidad}: vivir más de lo esperado.
\item
  Necesidad de \textbf{planificar a largo plazo}.
\end{itemize}
\end{frame}

\begin{frame}{Necesidades previsionales}
\phantomsection\label{necesidades-previsionales}
\begin{itemize}
\tightlist
\item
  Vivienda
\item
  Salud
\item
  Mantenimiento del nivel de vida
\item
  Apoyo a familiares
\end{itemize}
\end{frame}

\begin{frame}{Cálculo de la prestación pública esperada}
\phantomsection\label{cuxe1lculo-de-la-prestaciuxf3n-puxfablica-esperada}
\begin{itemize}
\tightlist
\item
  Basado en:

  \begin{itemize}
  \tightlist
  \item
    Años cotizados
  \item
    Base reguladora
  \item
    Edad de jubilación
  \end{itemize}
\item
  Realizar \textbf{proyecciones realistas}.
\end{itemize}
\end{frame}

\begin{frame}{Recursos previsionales complementarios}
\phantomsection\label{recursos-previsionales-complementarios}
\begin{itemize}
\tightlist
\item
  Planes de pensiones
\item
  Seguros de ahorro
\item
  Inversiones inmobiliarias
\item
  Carteras de inversión
\end{itemize}
\end{frame}

\begin{frame}{Presupuesto de jubilación}
\phantomsection\label{presupuesto-de-jubilaciuxf3n}
\begin{itemize}
\tightlist
\item
  Estimar gastos esperados vs.~ingresos previstos.
\item
  Detectar \textbf{déficits o superávits previsionales}.
\item
  Reajustar estrategia de ahorro o inversión.
\end{itemize}
\end{frame}

\begin{frame}{Evaluación de la situación financiera futura}
\phantomsection\label{evaluaciuxf3n-de-la-situaciuxf3n-financiera-futura}
\begin{itemize}
\tightlist
\item
  ¿Cuándo empezar?
\item
  ¿Cuánto ahorrar?
\item
  ¿Qué producto elegir?
\item
  ¿Cómo optimizar fiscalmente?
\end{itemize}
\end{frame}

\begin{frame}{Preguntas tipo test de repaso}
\phantomsection\label{preguntas-tipo-test-de-repaso}
\begin{tcolorbox}[enhanced jigsaw, rightrule=.15mm, colback=white, arc=.35mm, colframe=quarto-callout-note-color-frame, bottomrule=.15mm, left=2mm, toptitle=1mm, colbacktitle=quarto-callout-note-color!10!white, leftrule=.75mm, bottomtitle=1mm, titlerule=0mm, title=\textcolor{quarto-callout-note-color}{\faInfo}\hspace{0.5em}{Nota}, opacityback=0, coltitle=black, toprule=.15mm, opacitybacktitle=0.6, breakable]

\textbf{Ponderación de esta materia:} 5 \% del total del examen EFA NII.

\end{tcolorbox}
\end{frame}

\begin{frame}{Pregunta 1}
\phantomsection\label{pregunta-1}
¿Cuál de los siguientes elementos NO debe tenerse en cuenta para una
adecuada planificación financiera de la jubilación?

\begin{enumerate}
[a.]
\tightlist
\item
  Los gastos mensuales esperados a partir de la fecha de jubilación.
\item
  La inflación y los impuestos.
\item
  El número de descendientes del inversor.
\item
  La pensión pública de la Seguridad Social que se espera percibir a
  partir de la fecha de jubilación.
\end{enumerate}
\end{frame}

\begin{frame}{Respuesta 1}
\phantomsection\label{respuesta-1}
\begin{tcolorbox}[enhanced jigsaw, rightrule=.15mm, colback=white, arc=.35mm, colframe=quarto-callout-tip-color-frame, leftrule=.75mm, bottomrule=.15mm, left=2mm, toprule=.15mm, opacityback=0, breakable]
\begin{minipage}[t]{5.5mm}
\textcolor{quarto-callout-tip-color}{\faLightbulb}
\end{minipage}%
\begin{minipage}[t]{\textwidth - 5.5mm}

La respuesta \textbf{correcta es la c}.

Aunque el número de descendientes puede influir en otros aspectos
financieros (como herencias o dependencia económica), \textbf{no es un
factor clave} al elaborar un presupuesto para la jubilación centrado en
el equilibrio entre ingresos y gastos previsibles del propio jubilado.

\end{minipage}%
\end{tcolorbox}
\end{frame}

\begin{frame}{Pregunta 2}
\phantomsection\label{pregunta-2}
Calcule el GAP anual de ingresos de jubilación de un sujeto cuyo último
salario mensual, neto de impuestos, asciende a 4.000 euros, y que estima
percibir una pensión pública de jubilación, neta de impuestos, de 1.500
euros que se le abonará en 14 pagas.

\begin{enumerate}
[a.]
\tightlist
\item
  48.000
\item
  27.000
\item
  21.000
\item
  30.000
\end{enumerate}
\end{frame}

\begin{frame}{Respuesta 2}
\phantomsection\label{respuesta-2}
\begin{tcolorbox}[enhanced jigsaw, rightrule=.15mm, colback=white, arc=.35mm, colframe=quarto-callout-tip-color-frame, leftrule=.75mm, bottomrule=.15mm, left=2mm, toprule=.15mm, opacityback=0, breakable]
\begin{minipage}[t]{5.5mm}
\textcolor{quarto-callout-tip-color}{\faLightbulb}
\end{minipage}%
\begin{minipage}[t]{\textwidth - 5.5mm}

La respuesta \textbf{correcta es la b}.

\textbf{Paso 1}: Ingresos netos esperados en jubilación = 1.500 × 14 =
21.000 €/año\\
\textbf{Paso 2}: Último salario anual neto en activo = 4.000 × 12 =
48.000 €/año\\
\textbf{Paso 3}: GAP de ingresos = 48.000 − 21.000 = \textbf{27.000
€/año}

El GAP es la diferencia anual entre el nivel de ingresos deseado (último
salario) y la pensión pública esperada.

\end{minipage}%
\end{tcolorbox}
\end{frame}

\begin{frame}{Pregunta 3}
\phantomsection\label{pregunta-3}
¿Qué quiere decir que el sistema público de pensiones es un sistema de
reparto?

\begin{enumerate}
[a.]
\tightlist
\item
  Los rendimientos generados por la gestión pública de las cotizaciones
  sociales se atribuyen proporcionalmente a los pensionistas.
\item
  Las contribuciones de la población activa van dirigidas a pagar las
  pensiones de la población pasiva.
\item
  Las pensiones dependen de la rentabilidad obtenida por las
  cotizaciones.
\item
  Las cotizaciones se reparten entre comunidades autónomas según su
  número de pensionistas.
\end{enumerate}
\end{frame}

\begin{frame}{Respuesta 3}
\phantomsection\label{respuesta-3}
\begin{tcolorbox}[enhanced jigsaw, rightrule=.15mm, colback=white, arc=.35mm, colframe=quarto-callout-tip-color-frame, leftrule=.75mm, bottomrule=.15mm, left=2mm, toprule=.15mm, opacityback=0, breakable]
\begin{minipage}[t]{5.5mm}
\textcolor{quarto-callout-tip-color}{\faLightbulb}
\end{minipage}%
\begin{minipage}[t]{\textwidth - 5.5mm}

La respuesta \textbf{correcta es la b}.

El sistema público español se basa en \textbf{reparto
intergeneracional}: los cotizantes actuales financian las pensiones
actuales. \textbf{No hay capitalización individual ni acumulación de
fondos} para el futuro. Es un sistema de solidaridad directa.

\end{minipage}%
\end{tcolorbox}
\end{frame}

\begin{frame}{Pregunta 4}
\phantomsection\label{pregunta-4}
Un trabajador que actualmente cobra un sueldo mensual de 5.000 euros
espera recibir una pensión de jubilación de 1.500 euros. ¿Cuál será la
tasa de sustitución de su jubilación?

\begin{enumerate}
[a.]
\tightlist
\item
  Un 30\%.
\item
  Un 70\%.
\item
  Un 100\%.
\item
  Un 0\%.
\end{enumerate}
\end{frame}

\begin{frame}{Respuesta 4}
\phantomsection\label{respuesta-4}
\begin{tcolorbox}[enhanced jigsaw, rightrule=.15mm, colback=white, arc=.35mm, colframe=quarto-callout-tip-color-frame, leftrule=.75mm, bottomrule=.15mm, left=2mm, toprule=.15mm, opacityback=0, breakable]
\begin{minipage}[t]{5.5mm}
\textcolor{quarto-callout-tip-color}{\faLightbulb}
\end{minipage}%
\begin{minipage}[t]{\textwidth - 5.5mm}

La respuesta \textbf{correcta es la a}.

\textbf{Tasa de sustitución} = (1.500 / 5.000) × 100 = \textbf{30\%}

Esta tasa mide el porcentaje del último salario que representa la
pensión de jubilación.

\end{minipage}%
\end{tcolorbox}
\end{frame}

\begin{frame}{Pregunta 5}
\phantomsection\label{pregunta-5}
¿Qué se entiende por déficit anual del sistema público de pensiones?

\begin{enumerate}
[a.]
\tightlist
\item
  Rentabilidad negativa de fondos de pensiones.
\item
  Diferencia entre ingresos por cotizaciones y gasto en pensiones
  públicas.
\item
  Diferencia entre salario y pensión.
\item
  Exceso de aportaciones privadas sobre el límite anual.
\end{enumerate}
\end{frame}

\begin{frame}{Respuesta 5}
\phantomsection\label{respuesta-5}
\begin{tcolorbox}[enhanced jigsaw, rightrule=.15mm, colback=white, arc=.35mm, colframe=quarto-callout-tip-color-frame, leftrule=.75mm, bottomrule=.15mm, left=2mm, toprule=.15mm, opacityback=0, breakable]
\begin{minipage}[t]{5.5mm}
\textcolor{quarto-callout-tip-color}{\faLightbulb}
\end{minipage}%
\begin{minipage}[t]{\textwidth - 5.5mm}

La respuesta \textbf{correcta es la b}.

El déficit es el resultado de que \textbf{los ingresos por cotizaciones
sociales no cubren el gasto anual en pensiones}. Este saldo negativo se
financia mediante transferencias del Estado, lo que pone en entredicho
la \textbf{sostenibilidad del sistema}.

\end{minipage}%
\end{tcolorbox}
\end{frame}

\begin{frame}{Pregunta 6}
\phantomsection\label{pregunta-6}
Desde el 1 de abril de 2013, ¿de qué factores depende la edad de
jubilación?

\begin{enumerate}
[I.]
\tightlist
\item
  De la edad del trabajador.\\
\item
  De la tasa de inflación en la fecha en que se cumplan 65 años.\\
\item
  De las cotizaciones a la Seguridad Social acumuladas a lo largo de la
  vida laboral.\\
\item
  Del importe de las cotizaciones realizadas a la jubilación en el
  último año previo a la jubilación.
\end{enumerate}

\begin{enumerate}
[a.]
\tightlist
\item
  I, III y IV.
\item
  Sólo la I.
\item
  I y III.
\item
  Sólo la III.
\end{enumerate}
\end{frame}

\begin{frame}{Respuesta 6}
\phantomsection\label{respuesta-6}
\begin{tcolorbox}[enhanced jigsaw, rightrule=.15mm, colback=white, arc=.35mm, colframe=quarto-callout-tip-color-frame, leftrule=.75mm, bottomrule=.15mm, left=2mm, toprule=.15mm, opacityback=0, breakable]
\begin{minipage}[t]{5.5mm}
\textcolor{quarto-callout-tip-color}{\faLightbulb}
\end{minipage}%
\begin{minipage}[t]{\textwidth - 5.5mm}

La respuesta \textbf{correcta es la c}.

A partir del 1 de enero de 2013, la edad ordinaria de jubilación varía
entre los 65 y los 67 años, dependiendo de los \textbf{años cotizados}.
Si se han cotizado al menos 38 años y 6 meses, se puede jubilar a los
65; en caso contrario, la edad es 67 años. La \textbf{edad del
trabajador y los años cotizados acumulados} son los factores relevantes.

\end{minipage}%
\end{tcolorbox}
\end{frame}

\begin{frame}{Pregunta 7}
\phantomsection\label{pregunta-7}
AI preverse legalmente el pago de una cuota máxima de pensión pública,
¿quiénes tendrán más necesidad de realizar una planificación de su
jubilación?

I. Los clientes que tengan actualmente ingresos bajos pero tengan
acumulados muchos ahorros.\\
II. Los clientes que piensen jubilarse anticipadamente y no tengan
ahorros.\\
III. Los que tengan un número de años de cotización inferior al exigido
por la ley para obtener la pensión máxima de jubilación.\\
IV. Los que estimen unos ingresos netos post-jubilación superiores al
salario neto actual.

\begin{enumerate}
[a.]
\tightlist
\item
  II y III.
\item
  Sólo la III.
\item
  I, II y III.
\item
  Sólo la I.
\end{enumerate}
\end{frame}

\begin{frame}{Respuesta 7}
\phantomsection\label{respuesta-7}
\begin{tcolorbox}[enhanced jigsaw, rightrule=.15mm, colback=white, arc=.35mm, colframe=quarto-callout-tip-color-frame, leftrule=.75mm, bottomrule=.15mm, left=2mm, toprule=.15mm, opacityback=0, breakable]
\begin{minipage}[t]{5.5mm}
\textcolor{quarto-callout-tip-color}{\faLightbulb}
\end{minipage}%
\begin{minipage}[t]{\textwidth - 5.5mm}

La respuesta \textbf{correcta es la a}.

Los supuestos II y III reflejan una situación de \textbf{mayor
vulnerabilidad financiera}: sin ahorros y sin derecho a la pensión
máxima. En cambio, los clientes con ahorros acumulados o expectativas
poco realistas (IV) pueden tener margen de maniobra o simplemente no se
encuentran en una situación crítica.

\end{minipage}%
\end{tcolorbox}
\end{frame}

\begin{frame}{Pregunta 8}
\phantomsection\label{pregunta-8}
¿Qué es un PEPP (Producto Paneuropeo de Pensiones Individuales)?

\begin{enumerate}
[a.]
\tightlist
\item
  Un seguro colectivo obligatorio promovido por la Seguridad Social.
\item
  Un producto financiero solo disponible para empleados públicos.
\item
  Un instrumento de ahorro voluntario armonizado a nivel europeo.
\item
  Un fondo de inversión libre especializado en renta fija.
\end{enumerate}
\end{frame}

\begin{frame}{Respuesta 8}
\phantomsection\label{respuesta-8}
\begin{tcolorbox}[enhanced jigsaw, rightrule=.15mm, colback=white, arc=.35mm, colframe=quarto-callout-tip-color-frame, leftrule=.75mm, bottomrule=.15mm, left=2mm, toprule=.15mm, opacityback=0, breakable]
\begin{minipage}[t]{5.5mm}
\textcolor{quarto-callout-tip-color}{\faLightbulb}
\end{minipage}%
\begin{minipage}[t]{\textwidth - 5.5mm}

La respuesta \textbf{correcta es la c}.

El \textbf{PEPP} es un producto de ahorro voluntario para la jubilación,
regulado por la \textbf{UE}, con normas comunes para todos los Estados
miembros. Fomenta la \textbf{portabilidad} entre países, la
transparencia en costes y la posibilidad de elegir entre distintas
opciones de inversión.

\end{minipage}%
\end{tcolorbox}
\end{frame}

\begin{frame}{Pregunta 9}
\phantomsection\label{pregunta-9}
¿Qué modalidad de plan de pensiones se caracteriza por tener claramente
definidos los derechos de prestación para el beneficiario?

\begin{enumerate}
[a.]
\tightlist
\item
  Plan de aportación definida
\item
  Plan de prestación definida
\item
  Plan mixto
\item
  Plan garantizado
\end{enumerate}
\end{frame}

\begin{frame}{Respuesta 9}
\phantomsection\label{respuesta-9}
\begin{tcolorbox}[enhanced jigsaw, rightrule=.15mm, colback=white, arc=.35mm, colframe=quarto-callout-tip-color-frame, leftrule=.75mm, bottomrule=.15mm, left=2mm, toprule=.15mm, opacityback=0, breakable]
\begin{minipage}[t]{5.5mm}
\textcolor{quarto-callout-tip-color}{\faLightbulb}
\end{minipage}%
\begin{minipage}[t]{\textwidth - 5.5mm}

La respuesta \textbf{correcta es la b}.

En un \textbf{plan de prestación definida}, la cuantía de la prestación
(por ejemplo, un porcentaje del salario) está fijada desde el principio.
El promotor asume el riesgo de inversión. Es más frecuente en el ámbito
laboral o colectivo.

\end{minipage}%
\end{tcolorbox}
\end{frame}

\begin{frame}{Pregunta 10}
\phantomsection\label{pregunta-10}
¿Cuál es la contingencia principal cubierta por los planes de pensiones?

\begin{enumerate}
[a.]
\tightlist
\item
  Incapacidad temporal
\item
  Fallecimiento
\item
  Jubilación
\item
  Desempleo involuntario
\end{enumerate}
\end{frame}

\begin{frame}{Respuesta 10}
\phantomsection\label{respuesta-10}
\begin{tcolorbox}[enhanced jigsaw, rightrule=.15mm, colback=white, arc=.35mm, colframe=quarto-callout-tip-color-frame, leftrule=.75mm, bottomrule=.15mm, left=2mm, toprule=.15mm, opacityback=0, breakable]
\begin{minipage}[t]{5.5mm}
\textcolor{quarto-callout-tip-color}{\faLightbulb}
\end{minipage}%
\begin{minipage}[t]{\textwidth - 5.5mm}

La respuesta \textbf{correcta es la c}.

Aunque los planes de pensiones también cubren otras contingencias
(incapacidad, fallecimiento, dependencia), \textbf{la jubilación es la
contingencia principal} que justifica la naturaleza de estos productos
como instrumentos de previsión social.

\end{minipage}%
\end{tcolorbox}
\end{frame}

\begin{frame}{Pregunta 11}
\phantomsection\label{pregunta-11}
En la planificación para la jubilación, ¿es necesario prever la cuantía
teórica de la pensión de la Seguridad Social que se cobrará cuando
llegue el momento de la jubilación?

\begin{enumerate}
[a.]
\tightlist
\item
  No es necesario, pues hay mucha incertidumbre sobre la cuantía.\\
\item
  Sí es necesario, pues servirá para saber las necesidades de ahorro.\\
\item
  Sí es necesario, aunque no se puede, pues no se sabe cuánto se va a
  cotizar hasta el momento de la jubilación.\\
\item
  No es necesario, pues independientemente de la pensión de la Seguridad
  Social es importante ahorrar para cuando llegue la jubilación.
\end{enumerate}
\end{frame}

\begin{frame}{Respuesta 11}
\phantomsection\label{respuesta-11}
\begin{tcolorbox}[enhanced jigsaw, rightrule=.15mm, colback=white, arc=.35mm, colframe=quarto-callout-tip-color-frame, leftrule=.75mm, bottomrule=.15mm, left=2mm, toprule=.15mm, opacityback=0, breakable]
\begin{minipage}[t]{5.5mm}
\textcolor{quarto-callout-tip-color}{\faLightbulb}
\end{minipage}%
\begin{minipage}[t]{\textwidth - 5.5mm}

La respuesta \textbf{correcta es la c}.

En la planificación para la jubilación, \textbf{es necesario prever la
cuantía teórica de la pensión pública futura para estimar el ahorro
complementario necesario}, aunque es cierto que \textbf{no se puede
calcular con exactitud}, ya que dependerá de múltiples factores: años
cotizados, bases de cotización, edad de jubilación y normativa vigente.

\textbf{Principales variables que influyen en la pensión:}\\
Años cotizados y base reguladora.\\
Lagunas de cotización.\\
Cambios legislativos (edad ordinaria, factor de sostenibilidad,
revalorización).

\end{minipage}%
\end{tcolorbox}
\end{frame}

\begin{frame}{Pregunta 12}
\phantomsection\label{pregunta-12}
En la planificación financiera para la jubilación, ¿cuál de los
siguientes factores \textbf{no es esencial} al determinar cuánto dinero
será necesario para complementar la pensión pública?

\begin{enumerate}
[a.]
\tightlist
\item
  El salario actual y su tasa de crecimiento.\\
\item
  La tasa de interés libre de riesgo.\\
\item
  La pensión pública actual y su incremento anual esperado.\\
\item
  El tiempo que falta para la jubilación.
\end{enumerate}
\end{frame}

\begin{frame}{Respuesta 12}
\phantomsection\label{respuesta-12}
\begin{tcolorbox}[enhanced jigsaw, rightrule=.15mm, colback=white, arc=.35mm, colframe=quarto-callout-tip-color-frame, leftrule=.75mm, bottomrule=.15mm, left=2mm, toprule=.15mm, opacityback=0, breakable]
\begin{minipage}[t]{5.5mm}
\textcolor{quarto-callout-tip-color}{\faLightbulb}
\end{minipage}%
\begin{minipage}[t]{\textwidth - 5.5mm}

La respuesta \textbf{correcta es la b}.

La \textbf{tasa libre de riesgo} es un concepto financiero teórico útil
para valorar inversiones, pero \textbf{no es un factor determinante
directo} al estimar el capital necesario para la jubilación. En cambio,
el salario actual, el importe estimado de la pensión pública y el
horizonte temporal sí son \textbf{variables clave} para una correcta
planificación previsional.

\end{minipage}%
\end{tcolorbox}
\end{frame}

\begin{frame}{Pregunta 13}
\phantomsection\label{pregunta-13}
En el caso de un trabajador por cuenta ajena, ¿cuál de las siguientes
preguntas \textbf{no es esencial} al planificar para la jubilación?

\begin{enumerate}
[a.]
\tightlist
\item
  ¿Cuándo empezar a ahorrar?\\
\item
  ¿Cuánto dinero será necesario?\\
\item
  ¿Cuánto cotizar a la Seguridad Social?\\
\item
  ¿Cuándo espera jubilarse?
\end{enumerate}
\end{frame}

\begin{frame}{Respuesta 13}
\phantomsection\label{respuesta-13}
\begin{tcolorbox}[enhanced jigsaw, rightrule=.15mm, colback=white, arc=.35mm, colframe=quarto-callout-tip-color-frame, leftrule=.75mm, bottomrule=.15mm, left=2mm, toprule=.15mm, opacityback=0, breakable]
\begin{minipage}[t]{5.5mm}
\textcolor{quarto-callout-tip-color}{\faLightbulb}
\end{minipage}%
\begin{minipage}[t]{\textwidth - 5.5mm}

La respuesta \textbf{correcta es la c}.

\textbf{La cotización a la Seguridad Social no es una decisión
individual libre del trabajador}, ya que depende de su categoría
profesional, tipo de contrato y la normativa vigente. Las otras tres
preguntas son fundamentales para una planificación personal y flexible
de la jubilación.

\end{minipage}%
\end{tcolorbox}
\end{frame}

\begin{frame}{Pregunta 14}
\phantomsection\label{pregunta-14}
¿Cuál de las siguientes cuestiones tiene una \textbf{menor influencia
directa} en la pensión pública de jubilación?

\begin{enumerate}
[a.]
\tightlist
\item
  El período de cotización.\\
\item
  La base de cotización.\\
\item
  Los ingresos del trabajador en activo.\\
\item
  La base reguladora.
\end{enumerate}
\end{frame}

\begin{frame}{Respuesta 14}
\phantomsection\label{respuesta-14}
\begin{tcolorbox}[enhanced jigsaw, rightrule=.15mm, colback=white, arc=.35mm, colframe=quarto-callout-tip-color-frame, leftrule=.75mm, bottomrule=.15mm, left=2mm, toprule=.15mm, opacityback=0, breakable]
\begin{minipage}[t]{5.5mm}
\textcolor{quarto-callout-tip-color}{\faLightbulb}
\end{minipage}%
\begin{minipage}[t]{\textwidth - 5.5mm}

La respuesta \textbf{correcta es la c}.

\textbf{La pensión pública no se calcula directamente sobre los ingresos
brutos}, sino sobre las \textbf{bases de cotización} registradas en la
Seguridad Social. Estas pueden coincidir con los ingresos, pero no
siempre es así. Por tanto, \textbf{los ingresos no determinan
directamente la pensión}.

\end{minipage}%
\end{tcolorbox}
\end{frame}

\begin{frame}{Pregunta 15}
\phantomsection\label{pregunta-15}
En un Plan de Pensiones del sistema individual, el partícipe conoce
durante la fase de aportaciones:

\begin{enumerate}
[a.]
\tightlist
\item
  La forma en que se determinará la cantidad de la pensión de
  jubilación.\\
\item
  El importe que lleva acumulado en cada momento.\\
\item
  La cantidad que recibirá en el momento de la jubilación.\\
\item
  El importe que debe tener acumulado en el momento de la jubilación.
\end{enumerate}
\end{frame}

\begin{frame}{Respuesta 15}
\phantomsection\label{respuesta-15}
\begin{tcolorbox}[enhanced jigsaw, rightrule=.15mm, colback=white, arc=.35mm, colframe=quarto-callout-tip-color-frame, leftrule=.75mm, bottomrule=.15mm, left=2mm, toprule=.15mm, opacityback=0, breakable]
\begin{minipage}[t]{5.5mm}
\textcolor{quarto-callout-tip-color}{\faLightbulb}
\end{minipage}%
\begin{minipage}[t]{\textwidth - 5.5mm}

La respuesta \textbf{correcta es la b}.

En los planes de pensiones individuales, el partícipe \textbf{puede
consultar en todo momento el saldo acumulado (derechos consolidados)}.
Sin embargo, \textbf{no puede conocer con certeza la cuantía final}, ya
que esta dependerá de la evolución de las aportaciones y la rentabilidad
obtenida por el fondo de pensiones.

\end{minipage}%
\end{tcolorbox}
\end{frame}




\end{document}
